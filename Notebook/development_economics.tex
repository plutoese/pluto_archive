\documentclass[UTF8,14pt,blue,hyperref={CJKbookmarks=true}]{beamer}
\usepackage[space,noindent]{ctex}

\usetheme{Madrid}

\begin{document}

\title{发展经济学}
\subtitle{导论}
\author{张晨峰}
\institute{华东理工大学商学院}
\date{\today}

% frame 1
\begin{frame}
\titlepage
\end{frame}

% frame 2
\section{教材}
\begin{frame}{教材}
\begin{block}{主要教材}
发展经济学(姚洋,北京大学出版社)
\end{block}

\begin{block}{课外阅读}
\begin{itemize}
\item 发展经济学(第6版,波金斯等)
\item Handbook of Development Economics(Volume1-Volume5)
\end{itemize}
\end{block}
\end{frame}

% frame 3
\section{评分}
\begin{frame}{评分标准}
\begin{itemize}
\item 总分:考试(70\%) + 平时成绩(30\%) = 总分(100\%)
\item 平时成绩:出席(40\%) + 作业(30\%) + 课堂参与度(30\%) = 平时成绩(100\%)
\end{itemize}
\end{frame}

% frame 4
\section{课程大纲}
\begin{frame}{课程大纲}
\begin{enumerate}
\item 导论
\item 经济增长理论和应用
\item 人口和经济发展
\item 农业和经济发展
\item 工业化、城市化和经济发展
\item 国际贸易、外资利用和经济发展
\item 不平等和经济发展
\item 制度和经济发展
\item 可持续发展
\end{enumerate}
\end{frame}

% frame 5
\section{导论}
\begin{frame}{1 导论}
\begin{block}{主要内容}
\begin{itemize}
\item 对发展的认识
\item 世界的发展
\item 经济增长和经济发展
\end{itemize}
\end{block}
\end{frame}

% frame 6
\section{对发展的认识}
\begin{frame}{1.1 对发展的认识}
\begin{block}{讨论}
\begin{itemize}
\item 关于社会发展、区域和国家发展的认识和见解
\item 中国在发展中所面临的实际问题及解决方案
\end{itemize}
\end{block}
\end{frame}

% frame 7
\begin{frame}{1.2 世界的发展}
\includegraphics[scale=0.45]{perGDP.jpg}
\end{frame}

% frame 7
\begin{frame}{1.2 世界的发展}
\begin{block}{瘟疫创造的世界}
1346至1348年,由于热那亚商人,老鼠把跳蚤和瘟疫传播到了整个地中海地区。这种瘟疫几乎杀死了它所传播到地区一半的人口。在14世纪的转折点上,欧洲的社会形态是封建社会。黑死病造成的劳动力大量缺乏动摇了封建制度的基础,促使封建劳役逐渐消失,一个包容性的劳动力市场开始在英国出现,报酬也上涨了。
\end{block}
\end{frame}

% frame 8
\begin{frame}{1.2 世界的发展}
\begin{block}{禁止船运}
在宋朝,中国的许多技术发明都是世界领先的。中国人在欧洲人之前发明了钟表、指南针、火药、造纸术。因此,1500年中国的生活水平至少跟欧洲一样高。但是,中国专制的皇帝们反对变革,寻求稳定。国际贸易史就是对它最好的说明。洪武帝担心海外贸易会造成政治和社会不稳定,所以只允许由政府组织国际贸易。只有在永乐皇帝在位期间,郑和六次下西洋。然而,在此之后,所有的远洋贸易都被禁止了。
\end{block}
\end{frame}

% frame 9
\begin{frame}{1.3 经济增长和经济发展}
\begin{block}{经济增长}
一个国家总体或人际收入或产出的增长。
\end{block}
\begin{block}{经济发展}
经济发展的含义更加广泛,包括卫生、教育和人类其他福利的改进。发展同样伴随着经济结构的重大转变。
\end{block}
\end{frame}

% frame 10
\begin{frame}{1.3 经济增长和经济发展}
\begin{block}{经济增长的衡量}
国内生产总值(GDP)、国民生产总值(GNP)或者人均值。
\end{block}
\begin{block}{经济发展的衡量}
联合国开发计划署(UNDP)的人类发展指数(HDI)
\end{block}
\end{frame}

% frame 10
\begin{frame}{1.3 经济增长和经济发展}
\begin{block}{经济增长有用么?}
经济增长的用处并不在于财富提升了幸福感,而在于它扩大了人类选择的范围。财富和幸福感之间很难相关。幸福感源自一个人看待生活的方式......因此我们也就不能得出结论说财富的增加会使人们更幸福......经济增长的意义在于它给予人们更大的控制力,以控制其环境,并因此增加他的自由。
\end{block}
\end{frame}

\end{document}




