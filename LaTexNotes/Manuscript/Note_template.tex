% -*- coding:UTF-8 -*-
% a note template

% preamble
\documentclass[UTF8,a4paper,12pt]{article}
\usepackage[space,noindent]{ctex}
\usepackage{listings}

\title{题目}
\author{张晨峰}
\date{\today}

% body
\begin{document}
\maketitle

\section{笔记}

正确使用符号

使用破折号,我-们,我--们,我---们。
使用省略号,1,2,3\ldots;4,5,6\dots;7,8,9...
不能直接录入的标点符号,\# \quad \$ \quad \% \quad \& \quad \{ \quad \} \quad \_ \quad \textbackslash


空格与换行

任意多个空格     和一个空格 的功能相同。
单个换行也被看做是一个空格。
以字母命令的宏,后面的空格会被忽略,如果需要在命令后面使用空格,例如Happy \TeX\ ing;也可以在命令后面加一个空的分组{},例如Happy \TeX{} ing。
在空格中,最神奇的一种是被称为幻影(phantom)的空格,例如幻影\phantom{我是空格}速速隐形。
另起一行,并不分段,可以用\\命令,或者\linebreak命令。


字体

带参数的修改字体命令,用于少量字体的更换。
预定义命令的字体族有三种:\textrm{罗马},\textsf{无衬线},\texttt{打字机}。
预定义命令的字体形状有四种:\textup{直立},\textit{意大利},\textsl{倾斜},\textsc{small capitals shape}。
预定义命令的字体系列有两类:\textmd{中等},\textbf{加宽加粗}。
字体声明的修改字体命令,用于分组或环境中字体的整体更换。
\sffamily
\textnormal{相比较正常字体},这是无衬线的整体修改方案。
\rmfamily
中文字体
{\CJKfamily{zhhei}这是黑体},{\CJKfamily{zhkai}这是楷书},{\songti 宋体 \quad \heiti 黑体 \quad \fangsong 仿宋 \quad \kaishu 楷书}


强调文字

\rmfamily
文字可以\emph{强调}, 或者用\underline{下划线}。


字体大小

\noindent\tiny We \scriptsize start \footnotesize \small small,
\normalsize get \large big \Large and \LARGE bigger,\huge huge and \Huge gigantic!

\normalsize


文本环境

\begin{quote}
  这是引用环境。
\end{quote}

\begin{abstract}
  这是摘要环境。
\end{abstract}


列表环境

\begin{enumerate}
  \item 中文
  \item 英文
  \item 日文
\end{enumerate}

\begin{itemize}
  \item 中文
  \item 英文
  \item 日文
\end{itemize}


程序代码

\lstset{columns=flexible,numbers=left,numberstyle=\footnotesize}
\begin{lstlisting}[language=C]
/* hello.c */
#include <stdio.h>
main() {
    printf("hello.\n");
}
\end{lstlisting}


\end{document} 