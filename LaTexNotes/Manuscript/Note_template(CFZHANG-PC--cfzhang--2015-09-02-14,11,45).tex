% -*- coding:UTF-8 -*-
% a note template

% preamble
\documentclass[UTF8,a4paper,12pt]{article}
\usepackage[space,noindent]{ctex}

\title{题目}
\author{张晨峰}
\date{\today}

% body
\begin{document}
\maketitle

\section{文字和符号}

正确使用符号

使用破折号,我-们,我--们,我---们。
使用省略号,1,2,3\ldots;4,5,6\dots;7,8,9...
不能直接录入的标点符号,\# \quad \$ \quad \% \quad \& \quad \{ \quad \} \quad \_ \quad \textbackslash

空格与换行
任意多个空格     和一个空格 的功能相同。
单个换行也被看做是一个空格。
以字母命令的宏,后面的空格会被忽略,如果需要在命令后面使用空格,例如Happy \TeX\ ing;也可以在命令后面加一个空的分组{},例如Happy \TeX{} ing。
在空格中,最神奇的一种是被称为幻影(phantom)的空格,例如幻影\phantom{我是空格}速速隐形。
另起一行,并不分段,可以用\\命令,或者\linebreak命令。


文字可以\emph{强调}。文字可以是\textit{意大利形式},也可以是\textbf{粗体}的,\textsl{倾斜}的或者\textsc{capital}。
命令还可以是嵌套,比如\textit{\textbf{意大利粗体}}。网址可以用打印机表示,例如\texttt{http://www.ctan.org}。

\noindent\tiny We \scriptsize start \footnotesize \small small,
\normalsize get \large big \Large and \LARGE bigger,\huge huge and \Huge gigantic!

\normalsize Statement \#1:
50\% of \$100 makes \$50.
More special symbols are \&, \_, \{ and \}.

The hypotenuse: $\sqrt{a^{2} + b^{2}}$. I can type math!
\end{document} 